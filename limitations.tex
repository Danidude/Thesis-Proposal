\chapter{Limitations and key assumptions}
\label{ch:limitations}

%1-2 pages
%Fits into the method chapter in the thesis.
%Make it clear what you will not do.
%Make it clear what you need to assume to carry out your
%master thesis.

%Limitation: Example
%Applying NN algorithms to governmental web sites.
%Not personal web sites.
%Not business web sites.
%Modelling the development cycle in mobile software developments.
%Not application development.
%Not deployment.

%Assumptions: Example
%Assume that a web site can be modelled as a graph with
%nodes and vertices.
%Assume that wireless data can be modelled using statistics.
%Assume that useful data can be collected from the network
%using sniffer applications.

%Wrong assumptions
%Assume that I will answer my problem deffinitions. 
%Assume that I will prove my hypthesis.

To properly define the scope of our master thesis we have outlined
some limitations and assumptions. We will limit ourselves to testing and modifying algorithms;
we will not create completely new algorithms, a fully fleshed out system that would distribute the
safest paths to the passenger nor will we be creating the application that would run on the passenger's smartphones.
The focus in our master thesis will be on the algorithms themselves and not the final system as that have a large set
of challenges itself. For instance the complete system would need to find ways to distribute directions to each phone,
discover which phones are within the crisis area, receive information about the location of the fire onboard the ship etc.

We will apply current knowledge of human behavior during crisis. The field of human behavior during a crisis is large
and expanding it would be outside of our field of expertice. And as such we will use existing models and knowledge
to simulate the demeanor of the passengers. 

We will apply current knowledge on evacuation considering the occupant density around exits. There are studies
done in this field, not only theoretical but practical studies where people evacuate areas with high occupant density 
to test which exit is the optimal one \cite{Liu20091921}. Thus it is better to use this realisticly generated data than creating our own theoretical data.


We asume a ship can be sufficently modeled as a graph with nodes and vertices. 
We asume that human behavior in a crisis can be realisticly modeled and
predicted. We asume that hazards can be modeled.
