\chapter{Prior research:}
\label{ch:prior}
% 3-8 pages
% What research existed before you started.
% What research have you based your work on.
% What are the limitations of other peoples work.
% Doing exactly what others have done is not by itself research.
% You must do it dierent/better/...
% Your supervisor knows state of the art very well. S/he will judge you by it.

\section{Human Behavor Models}
There exists several behavioral models that predict movement. However the most successfully ones predict 
animal movement as animals are rather simple compared to humans. The environment for animals tends to 
be simpler, for instance ocean compared to urban, and animals are simpler creatures and will always rely only 
on instinct, whether in an emergency or not. There are several different computational tools used to simulate  
human movement during emergencies, for instance fluid and particle systems, matrix-based systems and  
emergent systems. Nevertheless they all either have limitations or they are somewhat inaccurate. 
 
It has been found that fluid and particle systems do not adequately describe human movement as humans  
panic and make choices, both good and bad, and they all have individual quirks. For instance, in a room with  
multiple exits a crowd tend to form a herd mentality and the majority will gather around a single exit, which  
will slow down their speed, and other exits will see little use. However had this been simulated with a fluid  
and particle system it would likely predict that all exits would be used equally for maximum effect. 
 
In a matrix-based system the room(s) and divided into cells. The cells can represent a table, it can be full of  
people or it can be empty. People can move from cell to cell at a rate described in the model and by the  
rules of the model. It is important that these cells and rules are properly adjusted to fit each model as to  
make it as realistic and precise as possible. 
 
In emergent systems simple parts can interact to simulate complex systems. The existing emergent systems  
are often criticized for being too simple. The system are often given only a few parameters (for example the  
average speed of the humans and the location of the exits) and then attempt to model the situation by  
assuming all humans move towards the most logical exit. 
 
In general the computational tools rely too much on assumptions and too little on the sociological and psychological  
factors. When a human is put into a crisis-situation that person can either rely on instinct, experience or making  
choices based on the alternatives. Relying on instinct typically means that the person is panicking and make immediate  
decisions that may or may not be good ones. People have been trampled to death by crowds that desperately is trying  
to get away from danger. 
 
A person will often follow the path it knows the best. If a worker walks up the same flight of stairs to his office  
every day for 5 years it is expected that he will also escape down the same flight of stairs during a crisis. This  
is why there are fire drills, so that during a crisis everyone have walked the quickest paths out. 
 
In a crisis that is time-sensitive, for instance a fire, it is very difficulty for a person to weigh his options and  
choose carefully. Some people will not stand around to observe which path might be the best or whether the  
door is hot before opening it. And standing around and choosing carefully might be even worse as  
over time there might be fewer safe paths as the fire spread. Therefore most people will make quick choices 
and move as fast as they can if danger is imminent. 
 
An individual will likely behave different when he is alone, as opposed to when he is in a group. For example  
a family will likely stick together and follow the leader, which is more often than not the father. Thus putting  
his or hers experience and instincts aside and becoming a part of the group. Emotions also have a tendency  
to spread quickly in a group, when some people panic it is likely that more will panic shortly thereafter. If  
there is widespread panic within the group the chance of people pushing down and trample others increase  
significantly. 
 
Xiaoshan Pan et al. created a framework to simulate human behavior. This framework implemented some 
basic human sociological and psychological factors. Though in this model the humans were not given a path 
to safety, they were merely left to their own devices. We will create a model where we give the people the 
safest path to the exit. We aim to have the people act as realistic as possible and be placed in a dynamic 
and complex environment.
 
 
\section{ACO}
Ant Colony Orginazation(ACO) have been around for quite some time, and there are sevral different applications this algoritm can be used for, for example routing, scheduling, subset and machine learning. The offical website to ACO is www.aco-metaheuristic.org. There are several different ACO systems that have been developed, two of the most successful are MAX-MIN Ant system and Ant Colony system.

In the orignial Ant system, each iteration runs m ants through the different paths/graph, and creates a solution. When all the m ants have finished, the update to pheromones happens to the solution the ants have found. Then the algorithm starts the next iteration with the next m ants.

MAX-MIN Ant system is an improvement to the original in the way that only the best solution is updated or best so far.

Ant Colony system uses “local pheromones” and “offline pheromones” in finding the best solution to the problem. The local pheromones are used to update each location so that the other ants that follows are less likely to take the same path as him, and thus creating several different solutions in the iteration. When the iteration is done, the offline pheromones are updated to the best solution making it more likely to select this one in the future.

