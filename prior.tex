\chapter{Prior research:}
\label{ch:prior}
\textit{3-8 pages}

There are several different computational tools used to simulate human movement during emergencies, 
for instance fluid and particle systems, matrix-based systems and emergent systems. 

It has been found that fluid and particle systems do not adequately describe human movement as humans 
panic and make choices, both good and bad, and they all have individual quirks. For instance, in a room with 
multiple exits a crowd tend to form a herd mentality and the majority will gather around a single exit, which 
will slow down their speed, and other exits will see little use. However had this been simulated with a fluid 
and particle system it would likely predict that all exits would be used equally for maximum effect.

In a matrix-based system the room(s) and divided into cells. The cells can represent a table, it can be full of 
people or it can be empty. People can move from cell to cell at a rate described in the model and by the 
rules of the model. It is important that these cells and rules are properly adjusted to fit each model as to 
make it as realistic and precise as possible.

In emergent systems simple parts can interact to simulate complex systems. The existing emergent systems 
are often criticized for being too simple. The system are often given only a few parameters (for example the 
average speed of the humans and the location of the exits) and then attempt to model the situation by 
asuming all humans move towards the most logical exit.

In general the computational tools rely too much on assumptions and too little on the sociological and psychological 
factors. When a human is put into a crisis-situation that person can either rely on instinct, experience or making 
choices based on the alternatives. Relying on instict typically means that the person is panicing and make immediate 
decisions that may or may not be good ones. People have been trambled to death by crowds that desperatly is trying 
to get away from the danger.

A person will often follow the path it knows the best. If a worker walks up the same flight of stairs to his office 
every day for 5 years it is expected that he will also escape down the same flight of stairs during a crisis. This 
is why there are fire drills, so that during a crisis everyone have walked the quickest paths out.

In a crisis that is time-sensitive, for instance a fire, it is very difficulty for a person to weigh his options and 
chose carefully. Some people will not stand around to observe which path might be the best or wether the 
door is hot before opening the door. However standing around and chosing carefully might be even worse as 
over time there might be fewer safe paths as the fire spread.  

An individual will likely behave different when he is alove, as opposed to when he is in a group. For example
 a family will likely stick together and follow the leader, which is more often than not the father. Thus putting 
his or hers experience and instincts aside and becoming a part of the group. Emotions also have a tendency
 to spread quickly in a group, when some people panic it is likely that more will panic shortly thereafter. If 
there is widespread panic within the group the chance of people pushing down and trample others increase 
significantly.