\chapter{Problem Statement and Hypothesis}
\label{ch:problem}

%1-3 pages
% Precisely define your problem.
% What exactly will you try to solve.
% Try to make it measurable.
% The examiner will ask "Did the student answer the problem definition?"
% Problem definition should never be vague.

Our goal is to evaluate several algorithms and determine which algorithm will provide us with the
best route between two points in dynamic graphs with nodes and vertices. Our hypothesis
is that Ant Colony Optimization (ACO) will perform better than conventional pathfinding
algorithms. Performance is measured by how many passengers survive the crisis situation
and how efficient the algorithms are. 

Secondly, we hypothesize that ACO will outperform djikstra's pathfinding algorithm
given a continuously changing graph. In a real crisis situation the environment is subject
to change. For instance a corridor that previously was safe can be filled with smoke
or a room that was empty can become filled with people and be undesirable as a path
to the exit.

Thirdly, we hypothesize that ACO will achieve a better outcome than djikstra's pathfinding algorithm
given high occupant density around exits. High occupant density is dangerous not only
because it will slow the evacuation speed it also creates potentially harmful situations. Thus the algorithm
should predict possible future bottlenecks and redirect passengers as needed.

Finally, we hypothesize that ACO will produce better results than djikstra's pathfinding algorithm
by adding human behavior. To produce accurate results the algorithms needs to be subjected
to a model that is as close to reality as possible and it then becomes necessary to include
human behavior to the model. Passengers can ignore directions, misunderstand directions, panic
and so forth. Behavior is different if the passenger is alone, with his or her family, or in a large group;
and thus it becomes important to account for individuals that may not follow the directions perfectly.

